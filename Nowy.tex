\documentclass{beamer}

\usepackage[polish]{babel}
\usepackage[T1]{fontenc}
\usepackage[utf8]{inputenc}
\usepackage{txfonts}

\usetheme{Warsaw}

\title{Problem  dystrybucji towarów z najwcześniejszymi i najpóźniejszymi terminami dostaw} 
\author{Dawid Ryznar, Krzystof Zielonka}

\date{24 październik 2012}

\begin{document}

\frame{\titlepage}

\begin{frame}{Problem  dystrybucji towarów z najwcześniejszymi i najpóźniejszymi terminami dostaw}

\begin{block}{Opis problemu}
\begin{itemize}
\item Pełny graf ważony z $n$ wierzchołkami,
\pause
\item Wyróżniony jeden wierzchołek $x$
\pause
\item Każdy wierzchołek ma $2$ funkcje,
\pause
\item $F:N \Rightarrow R$, gdzie $n$ - długość scieżki w momencie dotarcia do wierzchołka,
\pause
\item  $G:N \Rightarrow R$, gdzie $n$ - długość scieżki w momencie dotarcia do wierzchołka,
\end{itemize}
\end{block}
\end{frame}

\begin{frame}{Problem  dystrybucji towarów z najwcześniejszymi i najpóźniejszymi terminami dostaw}
\begin{block}{Cel}
\begin{itemize}
\item \textbf{Celem} jest znalezienie ścieżki startującej w $x$, która minimalizuje sumę wartości funkcji $F$ i $G$ oraz długość ścieżki,
\pause
\item Problem dystrybucji towarów z najwcześniejszymi i najpóźninejszymi terminami dostaw redukuje się do NP-zupełnego "Problemu Podziału na Podzbiory" \ [ang. \textit{SPP - Set Partitioning Problem}],
\end{itemize}
\end{block}
\end{frame}

\begin{frame}{Problem  dystrybucji towarów z najwcześniejszymi i najpóźniejszymi terminami dostaw}
\begin{block}{Przestrzeń poszukiwań}
\begin{itemize}
\item W celu znalezienia rozwiązania instancji problemu dystrybucji towarów z najwcześniejszymi i najpóźninejszymi terminami dostaw, musimy rozważać zbiory wszystkich możliwych scieżek zaczynających się w $x$, w pełnym grafie.
\pause
\item Dla grafu n wierzchołkowego mamy 
\pause
\item  tutaj jebnać trzeba wzór
\end{itemize}
\end{block}
\end{frame}

\end{document}