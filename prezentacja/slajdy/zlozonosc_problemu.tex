\begin{frame}{Złożonośc problemu}
	\begin{block}{Złożnosc problemu}
		Problem dystrupucji towarów jest NP trudny.
		Udowodnimy to konstrująć wielomianową redukcje problemu komiwojażera do problemu dystrybucji towarów.
	\end{block}

	\begin{block}{Problem komiwojażera}
		Mamy dany pełny ważony graf G. Rozwiązaniem problemu jest minimalny cykl Hamiltona na tym grafie.
	\end{block}
\end{frame}

\begin{frame}{Złożonośc problemu}
	\begin{block}{Dowód}
		
		\begin{algorithm}[H]
			\KwData{G -graf pełny wazony, w - funkcja wagowa}
			\KwResult{trojka $\left<T, p, t_{max}\right>$}
			$f$ -- funkcja przyporządkowywujaca wierzchołkom grafu kolejne liczby naturalne \\
			$g$ -- funkcja przyporządkowwywująca krawędziom kolejne liczby naturalne \\
			$p(v, t) = \lambda\left( v, t \right) \rightarrow \sum_{\{u, v\} \in E} (2^{g(\left\{u, v\right\})} \& t) \cdot w(\left\{ u, v \right\}) $ \\
			\ForAll{$v, u \in V$}{
				$T[f(v), f(u)] = 2^{g(\left\{v,u\right\})}$ \\
			}
			$t_{max} = \sum\limits_{i=2}^{|V|} w(f(v_{i-1}), f(v_i))$ \\
			\Return $\left< T, p , t_{max} \right> $
		\end{algorithm}
	
	\end{block}

\end{frame}
