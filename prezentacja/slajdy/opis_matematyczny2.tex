\begin{frame}
	
	\begin{block}{Rozwiązania dopuszczalne}
		Rozwiązaniem dopuszczalnym (sepłniającym warunki zadania) jest permutacja liczb $1, \cdots, n$.
	\end{block}

	\begin{block}{Uzasadnienie}
		\begin{itemize}
			\item W rozwiązniu musza znaleść sie wszystkie miasta. (definicja problemu)
			\item W rozwiązaniu miasta nie mogą sie powtarzać. (nierówność trójkąta
				+ założenie o nieskończonej ładowności ciężarówki
				+ możliwość czekania w dowolnym mieście)
			\item W rozwiazniu nie uwzgledniamy magazynu. (to jest punkt startowy i końcowy
				+ założenie o nieskończonej ładowności ciężarówki)	
		\end{itemize}
	\end{block}

\end{frame}