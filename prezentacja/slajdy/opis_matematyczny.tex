\begin{block}{Opis problemu}
	Przez $DT$ oznaczmy problem dystrybucji towarów:
	\begin{equation}
		DT = \left< V, w, p, t_{max}\right>
	\end{equation}
	%Graf:
	%\begin{equation}
	%	G = \left< V, E \right>
	%\end{equation}
	%\begin{equation}
	%	E = \left\{ \{v, u\} \ : \ v, u \in V \right\}
	%\end{equation}
	\begin{equation}
		t : V \rightarrow N
	\end{equation}
	\begin{equation}
		p : V \times N \rightarrow R
	\end{equation}
	\begin{description}
		\item[$w$] -- przyporządkowuje krawędziom wagi (czasy podróży)
		\item[$p$] -- funckja kary, dla danego wierzchołka i czasu przybcia zwraca kare w postaci liczby rzeczywistej
		\item[$t_{max}$] -- górne ograniczenie na czas potrzebny na pokonaniem dystansu
	\end{description}
\end{block}
