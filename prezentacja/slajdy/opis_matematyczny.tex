\begin{frame}{Opis problemu}
	
	\begin{equation}
		DT = \left< V, w, p, t_{max}\right>
	\end{equation}
	\begin{equation}
		t : V \rightarrow N
	\end{equation}
	\begin{equation}
		p : V \times N \rightarrow R
	\end{equation}
	\begin{description}
		\item[$w$] -- przyporządkowuje krawędziom wagi (czasy podróży)
		\item[$p$] -- funckja kary, dla danego wierzchołka i czasu przybcia zwraca kare w postaci liczby rzeczywistej
		\item[$t_{max}$] -- górne ograniczenie na czas potrzebny na pokonaniem dystansu
	\end{description}

\end{frame}

\begin{frame}
	
	\begin{block}{Instancja problemu}
		Miastom przyporządkowujmey kolejno numery $2, \cdots, n$.
		Dla uproszczenia będziemy zakładać, że magazyn zawsze ma numer $1$.
		Instancją problemu jest para:
		\begin{equation}
			P = \left< T, p \right>
		\end{equation}
	\end{block}
	
	\begin{block}{Macierz czasu przajazdów}		
		\begin{equation}
			T = \left[ t_{ij} \right]_{n \times n} 
		\end{equation}
		Kwadratowa maciarz gdzie element $t_{ij}$ to czas potrzebny na przejazd najszybszą drogą z miasta $i$ do $j$.
	\end{block}
	
	\begin{block}{Funkcja kary}
		\begin{equation}
			p : N \rightarrow N \rightarrow R 
		\end{equation}
		Funkcja kary, przyjmująca kolejno numer miasta, czas rozładunku i zwracająca karę w postaci liczby rzeczywistej.
	\end{block}

\end{frame}

\begin{frame}{Graf miast}
		
	W grafie miast, wierzchołki interpretujemy jako miasta, a wagi na krawędziach jako czasy potrzebne na przedostanie sie z jednego miasta do drugiego.
	Czas na krawędzi $\{u, v\}$ jest najszybszym czasem potrzebnym na przedostanie sie z miasta $u$ do $v$.
	\begin{itemize}
		\item Graf miast spełnia nierówność trójkąta
			\begin{equation}
				a < b + c \ \wedge \ b < a + c \ \wedge \ c < a + b
			\end{equation}
			\begin{equation}
				\forall_{a,b,c \in V} t(\left\{a, c\right\}) < t(\left\{a, b\right\}) + t(\left\{b, c\right\})
			\end{equation}
		\item Graf miast jest grafem pełnym
	\end{itemize}
	Jeżeli graf niespłnia jednego z powyższych założen to możemy go do takiego przekonwertować wywołujac dla każdego wierzchołka $BFS$.

\end{frame}
