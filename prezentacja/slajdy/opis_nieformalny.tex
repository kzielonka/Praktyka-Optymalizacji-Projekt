\begin{frame}
	\begin{block}{Opis problemu (orginalny)}
		Jako firma transportowa mamy dostarczyć towary do miast. 
		Każde miasto ma ustaloną kare za spóźnienie lub przybycie zawcześnie. 
		Przemieszczenia się między miastami trwa pewną liczbę czasu.
		Należy znalęść ciąg miast, dla którego ciężarówka odwiedza każde miasto i minimalizujący sumę kar jaką trzeba zapłacić za zbyt wczesne lub zbyt późne przybycie.
		%\begin{itemize}
		%	\item Należy dostarczyć towary do miast
		%	\item Każde miasto ma określone terminy, w których należy dostarczyć towar
		%	\item Przemieszczanie miedzy miastami zajmuje pewien czas
		%	\item Nidotrzymanie terminu wiąże się z otrzymaniem kary
		%	\item Rozwiązaniem problemu jest ciąg miast, który minimalizuje koszt dostarczenia towarów do wszystkich miast.
		%\end{itemize}
	\end{block}
\end{frame}

\begin{frame}
	\begin{block}{Nasze założenia}
		\begin{itemize}
			\item Ciężarówka ma nieskończoną pojemność (jest w stanie zabrać towary dla wszystkich miast).
			\item Dla każdego miasta mamy funkcje kary od czasu (ogólniejszy wariant kar).
			\item Czas jest w postaci liczby naturalnej (np liczba sekund).
			\item Znamy pewne górne ogranicznie czasowe, które okresla maksymalny czas przejazdu.
			\item Rozładunek i załadunek nie wymaga czasu.
			\item W kązdym miesic możemy czekać dowolną liczbe czasu znim wyładujemy towar.
			\item Znamy wierzchołek startowy (magazyn).
		\end{itemize}
	\end{block}
\end{frame}

\begin{frame}
	\begin{block}{Instancja problemu}
		Miastom przyporządkowujmey numery.
		Instancją problemu jest para:
		\begin{equation}
			P = \left< T, p \right>
		\end{equation}
	\end{block}
	\begin{block}{Macierz czasu przajazdów}		
		\begin{equation}
			T = \left[ t_{ij} \right]_{n \times n} 
		\end{equation}
		Kwadratowa maciarz gdzie element $t_{ij}$ to czas potrzebny na przejazd najszybszą drogą z miasta $i$ do $j$.
	\end{block}
	\begin{block}{Funkcja kary}
		\begin{equation}
			p : N \rightarrow N \rightarrow R 
		\end{equation}
		Funkcja kary, przyjmująca kolejno numer miasta, czas rozładunku i zwracająca karę w postaci liczby rzeczywistej.
	\end{block}
\end{frame}

