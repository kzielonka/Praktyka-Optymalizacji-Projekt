\begin{frame}
	\begin{block}{Opis problemu (orginalny)}
		Jako firma transportowa mamy dostarczyć towary do miast. 
		Każde miasto ma ustaloną kare za spóźnienie lub przybycie zawcześnie. 
		Przemieszczenia się między miastami trwa pewną liczbę czasu.
		Należy znalęść ciąg miast, dla którego ciężarówka odwiedza każde miasto i minimalizujący sumę kar jaką trzeba zapłacić za zbyt wczesne lub zbyt późne przybycie.
		%\begin{itemize}
		%	\item Należy dostarczyć towary do miast
		%	\item Każde miasto ma określone terminy, w których należy dostarczyć towar
		%	\item Przemieszczanie miedzy miastami zajmuje pewien czas
		%	\item Nidotrzymanie terminu wiąże się z otrzymaniem kary
		%	\item Rozwiązaniem problemu jest ciąg miast, który minimalizuje koszt dostarczenia towarów do wszystkich miast.
		%\end{itemize}
	\end{block}
\end{frame}

\begin{frame}
	\begin{block}{Nasze założenia}
		\begin{itemize}
			\item Ciężarówka ma nieskończoną pojemność.
			      Jest w stanie zabrać towary dla wszystkich miast.
			\item Dla każdego miasta mamy funkcje kary od czasu .
			      Ogólniejszy wariant kar.
			\item Czas jest w postaci liczby naturalnej 
			      np liczba sekund
			\item Znamy pewne górne ogranicznie czasowe, które określa maksymalny czas przejazdu.			   
			\item Rozładunek i załadunek nie wymaga czasu.
			\item W kązdym miesic możemy czekać dowolną liczbe czasu znim wyładujemy towar.
			\item Znamy wierzchołek startowy (magazyn).
		\end{itemize}
	\end{block}
\end{frame}

\begin{frame}
	\begin{block}{Graf miast}
		W grafie miast, wierzchołki interpretujemy jako miasta, a wagi na krawędziach jako czasy potrzebne na przedostanie sie z jednego miasta do drugiego.
		Czas na krawędzi $\{u, v\}$ jest najszybszym czasem potrzebnym na przedostanie sie z miasta $u$ do $v$.
		\begin{itemize}
			\item Graf miast spełnia nierówność trójkąta
				\begin{equation}
					a < b + c \ \wedge \ b < a + c \ \wedge \ c < a + b
				\end{equation}
				\begin{equation}
					\forall_{a,b,c \in V} t(\left\{a, c\right\}) < t(\left\{a, b\right\}) + t(\left\{b, c\right\})
				\end{equation}
			\item Graf miast jest grafem pełnym
		\end{itemize}
		Jeżeli graf niespłnia jednego z powyższych założen to możemy go do takiego przekonwertować wywołujac dla każdego wierzchołka $BFS$.
	\end{block}
\end{frame}

\begin{frame}
	\begin{block}{Instancja problemu}
		Miastom przyporządkowujmey kolejno numery $1, \cdots, n-1$.
		Dla uproszczenia będziemy zakładać, że magazyn zawsze ma numer $0$.
		Instancją problemu jest para:
		\begin{equation}
			P = \left< T, p \right>
		\end{equation}
	\end{block}
	\begin{block}{Macierz czasu przajazdów}		
		\begin{equation}
			T = \left[ t_{ij} \right]_{n \times n} 
		\end{equation}
		Kwadratowa maciarz gdzie element $t_{ij}$ to czas potrzebny na przejazd najszybszą drogą z miasta $i$ do $j$.
	\end{block}
	\begin{block}{Funkcja kary}
		\begin{equation}
			p : N \rightarrow N \rightarrow R 
		\end{equation}
		Funkcja kary, przyjmująca kolejno numer miasta, czas rozładunku i zwracająca karę w postaci liczby rzeczywistej.
	\end{block}
\end{frame}

\begin{frame}
	\begin{block}{Rozwiązania dopuszczalne}
		Rozwiązaniem dopuszczalnym (sepłniającym warunki zadania) jest permutacja liczb $1, \cdots, n$.
	\end{block}

	\begin{block}{Uzasadnienie}
		\begin{itemize}
			\item W rozwiązniu musza znaleść sie wszystkie miasta. (definicja problemu)
			\item W rozwiązaniu miasta nie mogą sie powtarzać. (nierówność trójkąta
				+ założenie o nieskończonej ładowności ciężarówki
				+ możliwość czekania w dowolnym mieście)
			\item W rozwiazniu nie uwzgledniamy magazynu. (to jest punkt startowy i końcowy
				+ założenie o nieskończonej ładowności ciężarówki)	
		\end{itemize}
	\end{block}
\end{frame}
