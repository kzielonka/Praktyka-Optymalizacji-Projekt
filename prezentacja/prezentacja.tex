\documentclass{beamer}

\usepackage[polish]{babel}
\usepackage[T1]{fontenc}
\usepackage[utf8]{inputenc}
\usepackage{txfonts}
\usepackage{amsmath}

\usetheme{Warsaw}

\title{Problem  dystrybucji towarów z najwcześniejszymi i najpóźniejszymi terminami dostaw} 
\author{Dawid Ryznar, Krzystof Zielonka}

\date{24 październik 2012}

\begin{document}

\frame{\titlepage}

\begin{frame}{Opis problemu (orginalny)}
	
	Jako firma transportowa mamy dostarczyć towary do miast. 
	Każde miasto ma ustaloną karę za spóźnienie lub przybycie zawcześnie. 
	Przemieszczenie się między miastami trwa pewną liczbę czasu.
	Należy znalęść ciąg miast, dla którego ciężarówka odwiedza każde miasto i minimalizujący sumę kar jaką trzeba zapłacić za zbyt wczesne lub zbyt późne przybycie.
	
\end{frame}

\begin{frame}{Nasze założenia}

	\begin{itemize}
		\item Ciężarówka ma nieskończoną pojemność. Jest w stanie zabrać towary dla wszystkich miast
		\item Dla każdego miasta mamy funkcje kary od czasu. Ogólniejszy wariant kar
		\item Czas jest w postaci liczby naturalnej np. liczba sekund
		\item Znamy pewne górne ogranicznie czasowe, które określa maksymalny czas przejazdu			   
		\item Rozładunek i załadunek nie wymaga czasu
		\item W kązdym miesic możemy czekać dowolną liczbe czasu znim wyładujemy towar
		\item Znamy wierzchołek startowy (magazyn)
	\end{itemize}

\end{frame}

\begin{block}{Funkcja celu}
	\begin{equation}
		C : V^n \times \left\{ V \right\} \times R \rightarrow R
	\end{equation}

	\begin{equation}
		C(v, \left\{ v \right\}, t) = 0
	\end{equation}

	\begin{equation}
		C(v, \empty, t) = p(v, t)
	\end{equation}

	\begin{equation}
		C(v_1, \cdots, v_n, U, t) = \begin{cases}
			C(v_2, \cdots, v_n, U \cup \left\{ v_1 \right\}, t + w(\left\{v_1, v_2\right\}) \\ 
			\	\	\	\text{gdy} \ v_1 \in U \\
			C(v_2, \cdots, v_n, U, t + w(\left\{v_1, v_2\right\})) + p(v_1, t) \\
			\	\	\	\text{gdy} \ v_1 \not\in U \\
		\end{cases}
	\end{equation}
\end{block}


%\begin{frame}{Problem  dystrybucji towarów z najwcześniejszymi i najpóźniejszymi terminami dostaw}
%	\begin{block}{Opis problemu}
\begin{itemize}
\item Pełny graf ważony z $n$ wierzchołkami,
\pause
\item Wyróżniony jeden wierzchołek startowy $x$
\pause
\item Każdy wierzchołek ma funckje kary $p$ jaką trzeba zapłacić za przebycie w czasie $t$ (drobna modyfikacja pierwotnego problemu gdzie każdy wierzchołek miał minimalny czas przybycia i maksymalny czas przybcia)
\pause
\item Każda krawędź ma przyparzadkowany czas potrzebny na jej pokonanie.
\end{itemize}
\end{block}

%\end{frame}

%\begin{frame}{Problem dystrybucji towarów z najwcześniejszymi i najpóźniejszymi terminami dostaw}
%	\begin{block}{Opis problemu}
	\begin{equation}
		G = \left< V, E, w, p \right>
	\end{equation}
	\begin{equation}
		E = \left\{ \{v, u\} \ : \ v, u \in V \right\}
	\end{equation}
	\begin{equation}
		w : V \rightarrow R
	\end{equation}
	\begin{equation}
		p : V \times R \rightarrow R
	\end{equation}
	\begin{description}
		\item[$w$] -- przyporządkowuje krawędziom wagi (czasy podróży)
		\item[$p$] -- funckja kary, dla danego wierzchołka i czasu przybcia zwraca kare w postaci liczby rzeczywistej
	\end{description}
\end{block}

%\end{frame}

%\begin{frame}
%	\begin{block}{Nierówność trójkąta}
	\begin{equation}
		a < b + c \ \wedge \ b < a + c \ \wedge \ c < a + b
	\end{equation}
	\begin{itemize}
		\item Zakładamy, że dany czas przejazdu między dowolnymi dwoma miastami to średni czas potrzebny na pokonanie najszybszej trasy łączącej te dwa miasta.
		\item Dzięki temu założeniu graf dla miast spełnia nierówność trójkąta czyli:
			\begin{equation}
				\forall_{a,b,c \in V} t(\left\{a, c\right\}) < t(\left\{a, b\right\}) + t(\left\{b, c\right\})
			\end{equation}
		\item Eleminujemy możliwość rozwiązania gdzie miasta mogą sie powtarzać, jeżeli mamy dostarczyć towar do miasta $b$, a znajdujemy sie w mieścia $a$ to najszybsza droga między tymi miastami zajmuje $t(\left\{a, b\right\})$.
	\end{itemize}

	
\end{block}

%\end{frame}

%\begin{frame}{Problem  dystrybucji towarów z najwcześniejszymi i najpóźniejszymi terminami dostaw}
%	\begin{block}{Cel}
\begin{itemize}
\item \textbf{Celem} jest znalezienie ścieżki startującej w $x$, która minimalizuje sumę wartości funkcji $F$ i $G$ oraz długość ścieżki,
\pause
\item Problem dystrybucji towarów z najwcześniejszymi i najpóźninejszymi terminami dostaw redukuje się do NP-zupełnego "Problemu Podziału na Podzbiory" \ [ang. \textit{SPP - Set Partitioning Problem}],
\end{itemize}
\end{block}

%\end{frame}

%\begin{frame}
%	\begin{block}{Rozwiązanie}
	* Rozwiązanie zawiera wszystkie wierzchołki.
	** Rozwiązanie nie zawiera cykli:
	\begin{itemize}
		\item Krawędzie spełniają nierówność trójkąta
		\item Przed rozładunkiem może przeczekać w danym mieście dowolny okres czasu nie płacąc żadnej kary
	\end{itemize}
	Rozwiązaniem jest permutacja wierzchołków z $V$.
\end{block}

%\end{frame}

%\begin{frame}
%	%\input{./slajdy/czas_przejazdu.tex}
%\end{frame}

%\begin{frame}{Problem  dystrybucji towarów z najwcześniejszymi i najpóźniejszymi terminami dostaw}
%	\begin{block}{Funkcja celu}
	\begin{equation}
		C : V^n \times \left\{ V \right\} \times R \rightarrow R
	\end{equation}

	\begin{equation}
		C(v, \left\{ v \right\}, t) = 0
	\end{equation}

	\begin{equation}
		C(v, \empty, t) = p(v, t)
	\end{equation}

	\begin{equation}
		C(v_1, \cdots, v_n, U, t) = \begin{cases}
			C(v_2, \cdots, v_n, U \cup \left\{ v_1 \right\}, t + w(\left\{v_1, v_2\right\}) \\ 
			\	\	\	\text{gdy} \ v_1 \in U \\
			C(v_2, \cdots, v_n, U, t + w(\left\{v_1, v_2\right\})) + p(v_1, t) \\
			\	\	\	\text{gdy} \ v_1 \not\in U \\
		\end{cases}
	\end{equation}
\end{block}

%\end{frame}

%\begin{frame}{Problem  dystrybucji towarów z najwcześniejszymi i najpóźniejszymi terminami dostaw}
%	 \begin{block}{Przestrzeń poszukiwań}
\begin{itemize}
\item W celu znalezienia rozwiązania instancji problemu dystrybucji towarów z najwcześniejszymi i najpóźninejszymi terminami dostaw, musimy rozważać zbiory wszystkich możliwych scieżek zaczynających się w $x$, w pełnym grafie.
\pause
\item Dla grafu n wierzchołkowego mamy 
\pause
\item  tutaj jebnać trzeba wzór
\end{itemize}
\end{block}

%\end{frame}

\end{document}
