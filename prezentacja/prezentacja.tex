\documentclass{beamer}

\usepackage[polish]{babel}
\usepackage[T1]{fontenc}
\usepackage[utf8]{inputenc}
\usepackage{txfonts}
\usepackage{amsmath}

\usetheme{Warsaw}

\title{Problem  dystrybucji towarów z najwcześniejszymi i najpóźniejszymi terminami dostaw} 
\author{Dawid Ryznar, Krzystof Zielonka}

\date{24 październik 2012}

\begin{document}

\frame{\titlepage}

\begin{frame}
	\begin{block}{Opis problemu (orginalny)}
		Jako firma transportowa mamy dostarczyć towary do miast. 
		Każde miasto ma ustaloną kare za spóźnienie lub przybycie zawcześnie. 
		Przemieszczenia się między miastami trwa pewną liczbę czasu.
		Należy znalęść ciąg miast, dla którego ciężarówka odwiedza każde miasto i minimalizujący sumę kar jaką trzeba zapłacić za zbyt wczesne lub zbyt późne przybycie.
		%\begin{itemize}
		%	\item Należy dostarczyć towary do miast
		%	\item Każde miasto ma określone terminy, w których należy dostarczyć towar
		%	\item Przemieszczanie miedzy miastami zajmuje pewien czas
		%	\item Nidotrzymanie terminu wiąże się z otrzymaniem kary
		%	\item Rozwiązaniem problemu jest ciąg miast, który minimalizuje koszt dostarczenia towarów do wszystkich miast.
		%\end{itemize}
	\end{block}
\end{frame}

\begin{frame}
	\begin{block}{Nasze założenia}
		\begin{itemize}
			\item Ciężarówka ma nieskończoną pojemność (jest w stanie zabrać towary dla wszystkich miast).
			\item Dla każdego miasta mamy funkcje kary od czasu (ogólniejszy wariant kar).
			\item Czas jest w postaci liczby naturalnej (np liczba sekund).
			\item Znamy pewne górne ogranicznie czasowe, które okresla maksymalny czas przejazdu.
			\item Rozładunek i załadunek nie wymaga czasu.
			\item W kązdym miesic możemy czekać dowolną liczbe czasu znim wyładujemy towar.
			\item Znamy wierzchołek startowy (magazyn).
		\end{itemize}
	\end{block}
\end{frame}

\begin{frame}
	\begin{block}{Instancja problemu}
		Miastom przyporządkowujmey numery.
		Instancją problemu jest para:
		\begin{equation}
			P = \left< T, p \right>
		\end{equation}
	\end{block}
	\begin{block}{Macierz czasu przajazdów}		
		\begin{equation}
			T = \left[ t_{ij} \right]_{n \times n} 
		\end{equation}
		Kwadratowa maciarz gdzie element $t_{ij}$ to czas potrzebny na przejazd najszybszą drogą z miasta $i$ do $j$.
	\end{block}
	\begin{block}{Funkcja kary}
		\begin{equation}
			p : N \rightarrow N \rightarrow R 
		\end{equation}
		Funkcja kary, przyjmująca kolejno numer miasta, czas rozładunku i zwracająca karę w postaci liczby rzeczywistej.
	\end{block}
\end{frame}



\begin{frame}{Funkcja celu}
		
		\begin{equation}
			F : N^n \rightarrow R
		\end{equation}
		\begin{equation}
			F(v_1 \cdots v_n) = C(v_1, \cdots V_n, 0)
		\end{equation}

\end{frame}

\begin{frame}{Funkcja pomocnicza}
		
		\begin{equation}
			C : N^m \times N \rightarrow R \ \ \ \text{gdzie} \ m > 0
		\end{equation}
		\begin{equation}
			C(v, t) = \begin{cases}
				p(v, t)	&	\text{gdy} \ t < t_{max} 	\\
				+inf	&	\text{wpp}			\\
			\end{cases}
		\end{equation}
		\begin{equation}
			C(v_1, \cdots, v_n, t) = \min\limits_{t \leq t_c \le t_{max}}\left\{ C(v_2, \cdots, v_n, t_c + t_{1,2}) + p(v_1, t_c) \right\}
		\end{equation}

\end{frame}

%\begin{frame}{Problem  dystrybucji towarów z najwcześniejszymi i najpóźniejszymi terminami dostaw}
%	\begin{block}{Opis problemu}
\begin{itemize}
	\item Pełny graf ważony z $n$ wierzchołkami,
	\item Wyróżniony jeden wierzchołek startowy $v_{start}$
	\item Wprowadzmy funckje kary $p$ jaką trzeba zapłacić za dostarczenie towaru w czasie $t$ do pewnego miasta (bardziej ogólny wariant, miasta mogą nadawać kary bardziej swobodnie oraz mogą nadawać nagorody)
	\item Każda krawędź ma przyparzadkowany czas potrzebny na jej pokonanie.
	\item Dodatkowo zakładamy, że w każdym mieście możemy przeczekać pewien okres czasu.
	\item Dla uproszczenie zakładamy, że jest pewne górne ograniczenie na czas potrzebny na pokonaniem trasy. Jeżeli rozwiązanie potrzebuje więcej czasu zakładamy, że jest ono nieakceptowalne.
\end{itemize}
\end{block}

%\end{frame}

%\begin{frame}{Problem dystrybucji towarów z najwcześniejszymi i najpóźniejszymi terminami dostaw}
%	\begin{frame}{Opis problemu}
	
	\begin{equation}
		DT = \left< V, w, p, t_{max}\right>
	\end{equation}
	\begin{equation}
		t : V \rightarrow N
	\end{equation}
	\begin{equation}
		p : V \times N \rightarrow R
	\end{equation}
	\begin{description}
		\item[$w$] -- przyporządkowuje krawędziom wagi (czasy podróży)
		\item[$p$] -- funckja kary, dla danego wierzchołka i czasu przybcia zwraca kare w postaci liczby rzeczywistej
		\item[$t_{max}$] -- górne ograniczenie na czas potrzebny na pokonaniem dystansu
	\end{description}

\end{frame}

\begin{frame}
	
	\begin{block}{Instancja problemu}
		Miastom przyporządkowujmey kolejno numery $2, \cdots, n$.
		Dla uproszczenia będziemy zakładać, że magazyn zawsze ma numer $1$.
		Instancją problemu jest para:
		\begin{equation}
			P = \left< T, p \right>
		\end{equation}
	\end{block}
	
	\begin{block}{Macierz czasu przajazdów}		
		\begin{equation}
			T = \left[ t_{ij} \right]_{n \times n} 
		\end{equation}
		Kwadratowa maciarz gdzie element $t_{ij}$ to czas potrzebny na przejazd najszybszą drogą z miasta $i$ do $j$.
	\end{block}
	
	\begin{block}{Funkcja kary}
		\begin{equation}
			p : N \rightarrow N \rightarrow R 
		\end{equation}
		Funkcja kary, przyjmująca kolejno numer miasta, czas rozładunku i zwracająca karę w postaci liczby rzeczywistej.
	\end{block}

\end{frame}

\begin{frame}{Graf miast}
		
	W grafie miast, wierzchołki interpretujemy jako miasta, a wagi na krawędziach jako czasy potrzebne na przedostanie sie z jednego miasta do drugiego.
	Czas na krawędzi $\{u, v\}$ jest najszybszym czasem potrzebnym na przedostanie sie z miasta $u$ do $v$.
	\begin{itemize}
		\item Graf miast spełnia nierówność trójkąta
			\begin{equation}
				a < b + c \ \wedge \ b < a + c \ \wedge \ c < a + b
			\end{equation}
			\begin{equation}
				\forall_{a,b,c \in V} t(\left\{a, c\right\}) < t(\left\{a, b\right\}) + t(\left\{b, c\right\})
			\end{equation}
		\item Graf miast jest grafem pełnym
	\end{itemize}
	Jeżeli graf niespłnia jednego z powyższych założen to możemy go do takiego przekonwertować wywołujac dla każdego wierzchołka $BFS$.

\end{frame}

%\end{frame}

%\begin{frame}
%	\begin{frame}{Nierówność trójkąta}
	
	\begin{equation}
		a < b + c \ \wedge \ b < a + c \ \wedge \ c < a + b
	\end{equation}
	\begin{itemize}
		\item Zakładamy, że dany czas przejazdu między dowolnymi dwoma miastami to średni czas potrzebny na pokonanie najszybszej trasy łączącej te dwa miasta
		\item Dzięki temu założeniu graf dla miast spełnia nierówność trójkąta czyli:
			\begin{equation}
				\forall_{a,b,c \in V} t(\left\{a, c\right\}) < t(\left\{a, b\right\}) + t(\left\{b, c\right\})
			\end{equation}
		\item Eleminujemy możliwość rozwiązania gdzie miasta mogą sie powtarzać, jeżeli mamy dostarczyć towar do miasta $b$, a znajdujemy sie w mieścia $a$ to najszybsza droga między tymi miastami zajmuje $t(\left\{a, b\right\})$
	\end{itemize}

\end{frame}

%\end{frame}

%\begin{frame}{Problem  dystrybucji towarów z najwcześniejszymi i najpóźniejszymi terminami dostaw}
%	\begin{block}{Cel}
\begin{itemize}
\item \textbf{Celem} jest znalezienie ścieżki startującej w $x$, która minimalizuje sumę wartości funkcji $F$ i $G$ oraz długość ścieżki,
\pause
\item Problem dystrybucji towarów z najwcześniejszymi i najpóźninejszymi terminami dostaw redukuje się do NP-zupełnego "Problemu Podziału na Podzbiory" \ [ang. \textit{SPP - Set Partitioning Problem}],
\end{itemize}
\end{block}

%\end{frame}

%\begin{frame}
%	\begin{block}{Rozwiązanie}
	* Rozwiązanie zawiera wszystkie wierzchołki.
	** Rozwiązanie nie zawiera cykli:
	\begin{itemize}
		\item Krawędzie spełniają nierówność trójkąta
		\item Przed rozładunkiem może przeczekać w danym mieście dowolny okres czasu nie płacąc żadnej kary
	\end{itemize}
	Rozwiązaniem jest permutacja wierzchołków z $V$.
\end{block}

%\end{frame}

%\begin{frame}
%	%\input{./slajdy/czas_przejazdu.tex}
%\end{frame}

%\begin{frame}{Problem  dystrybucji towarów z najwcześniejszymi i najpóźniejszymi terminami dostaw}
%	\begin{frame}{Funkcja celu}
		
		\begin{equation}
			F : N^n \rightarrow R
		\end{equation}
		\begin{equation}
			F(v_1 \cdots v_n) = C(v_1, \cdots V_n, 0)
		\end{equation}

\end{frame}

\begin{frame}{Funkcja pomocnicza}
		
		\begin{equation}
			C : N^m \times N \rightarrow R \ \ \ \text{gdzie} \ m > 0
		\end{equation}
		\begin{equation}
			C(v, t) = \begin{cases}
				p(v, t)	&	\text{gdy} \ t < t_{max} 	\\
				+inf	&	\text{wpp}			\\
			\end{cases}
		\end{equation}
		\begin{equation}
			C(v_1, \cdots, v_n, t) = \min\limits_{t \leq t_c \le t_{max}}\left\{ C(v_2, \cdots, v_n, t_c + t_{1,2}) + p(v_1, t_c) \right\}
		\end{equation}

\end{frame}
%\end{frame}

%\begin{frame}{Problem  dystrybucji towarów z najwcześniejszymi i najpóźniejszymi terminami dostaw}
%	 \begin{block}{Przestrzeń poszukiwań}
\begin{itemize}
\item W celu znalezienia rozwiązania instancji problemu dystrybucji towarów z najwcześniejszymi i najpóźninejszymi terminami dostaw, musimy rozważać zbiory wszystkich możliwych scieżek zaczynających się w $x$, w pełnym grafie.
\pause
\item Dla grafu n wierzchołkowego mamy 
\pause
\item  tutaj jebnać trzeba wzór
\end{itemize}
\end{block}

%\end{frame}

\end{document}
